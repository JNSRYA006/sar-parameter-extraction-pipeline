%---------------------------------------------------------------------
%	Abstract
%---------------------------------------------------------------------
\chapter{Abstract}				
\label{ch:abs}
%---------------------------------------------------------------------
% TAKEN FROM UCT WEBSITE
% The Doctoral Degrees Board recommends that candidates include an abstract 
% that fits onto one page, and which includes the author's full name, thesis 
% title and date. The text should not exceed 350 words. Candidates may also 
% include a more substantial summary of their work, in addition to the abstract.
% Follow the same format for an MSc. abstract.

% The abstract should stand on its own. It should answer the following questions:
%	1.	What did the author do? What ideas, notions, hypotheses, 
%		concepts, theories or thoughts were investigated?
%	2.	How did the author do the work? What data were generated and used? 
%		What was the origin of the data? How were data gathered? 
%		What tests, scales, indices, or summary measures were used? 
%		In other words, how was the analysis and/or synthesis done?
%	3.	What were the conclusions and what were the significant findings?
%
% Some studies cannot readily be summarised in this way and require more 
% descriptive abstracts. Do not use telegraphic phrases. Do not repeat 
% information given in the title. Do not use abbreviations. The purpose of 
% an abstract is to enable a researcher/examiner to understand the essential 
% hypothesis, method and findings of the research.
%---------------------------------------------------------------------
\begin{center}
	\textbf{\Large \titl}\\
			\vskip 0.2cm
			\auth\\
			\vskip 0.2cm
	\textit{\footnotesize\today}
			\vskip 1cm
\end{center}
%---------------------------------------------------------------------
% Write abstract here [350 words]
%---------------------------------------------------------------------
This report details an interim investigation into sea ice parameter extraction from \acf{sar} data. This report forms part of the wave parameter extraction process required to apply attenuation and dispersion models onto for sea ice parameter extraction. The application of this report is to design a wave parameter extraction pipeline, for use as part of the larger Antarctic research being conducted at the University of Cape Town. This report details this process through the use of the \acf{hh} inversion procedure using Sentinel-1A \acf{sar} data. An investigation into the obtained results from this procedure was compared to \acf{csir} data obtained from a directional wave buoy located at Cape Point. 