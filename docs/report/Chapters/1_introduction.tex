%****************************************************
%	CHAPTER 1 - INTRODUCTION
%****************************************************
\chapter{Introduction}
\label{ch:intro}

%This project report details the implementation of a MATLAB pipeline for the processing of \ac{sar} data for parameter extraction. The desired application of this pipeline is part of larger Antarctic research projects and will allow characteristics of sea ice floes to be extracted from \acs{sar} data. This report consists of research into parameter extraction using \acs{sar} - particularly in an Antarctic context, the design and implementation of a pipeline to extract desired features from these data, and testing and validation steps taken to ensure the correctness of this pipeline. This introductory chapter aims to provide context to the background for the project as well as frame the project's objectives, scope, limitations and project development.




This project report outlines the implementation of a \textsc{matlab} pipeline designed for processing \ac{sar} data to extract essential parameters. The intended application of this pipeline is within broader Antarctic research initiatives, enabling the extraction of sea ice floe characteristics from \acs{sar} data. The report encompasses an exploration of wave parameter extraction using \acs{sar}. It introduces the pipeline's design and implementation for extracting the desired characteristics from these data. Furthermore, it covers the testing and validation procedures undertaken to ensure the pipeline's accuracy.

The primary goal of this introductory chapter is to provide contextual background for the project, framing its objectives, scope, limitations, and developmental stages.
%====================================================
\section{Background}
\label{sec:intro.background}
%====================================================


%====================================================
\section{Problem statement}
\label{sec:intro.problemstatement}
There is currently a lack of accurate measurement data for the \ac{miz}, particularly during the winter season. This is primarily due to the remote and harsh conditions of the Antarctic continent and the surrounding \ac{so}. Understanding sea ice in the \acs{miz} is an essential aspect of investigating global climate as it plays a pivotal role in storing and transporting heat and greenhouse gases. Due to the geographical and environmental differences between the Arctic and the Antarctic, the application of remote sensing techniques in the Arctic does not directly translate to the Antarctic. As a result, there has been poor characterisation of sea-ice ocean interactions, ice floe size, and shape – all of which are important factors when considering the impact of Antarctica on the global climate. To enhance the characterisation of these features remotely, parameter extraction algorithms need to be developed, analysed, and compared to a small set of in-situ measurements. This report aims to develop a pipeline for parameter extraction using \ac{sar} images, which will aid in understanding Antarctic sea ice characteristics, particularly the size, shape, and thickness of pancake ice in the \acs{miz}.
%====================================================
%====================================================
\section{Project objectives}
\label{sec:intro.projectObjectives}
%====================================================
This project aims to develop a \acs{sar} processing pipeline for parameter extraction using wave modelling techniques that can be tested and analysed. The pipeline will be initially tested using \acs{sar} data with known ground truths to establish its efficacy. To achieve this, wave data from the \ac{csir} will serve as the ground truth when analysing \acs{sar} data collected over regions of South Africa for wave parameter extraction. This approach will allow the system's performance to be validated before being tested on \acs{sar} data captures from the \href{https://www.sanap.ac.za/scale-winter-cruise-2022}{\ac{scale} 2022} cruise. The project objectives, as outlined in the project brief, are presented in Table \ref{tab:intro.projectObjectives.brief}.

\begin{table}[H]
\centering
\begin{tabular}{|l|}
\hline
\begin{tabular}[c]{@{}l@{}}(a) Understand the requirements of the project.\\    (b) Conduct a literature review of previous work in this field and \\ critically evaluate current technology/research.\\    (c) Design \acs{sar} pre-processing pipelines for \acs{sar} data and \\ develop parameter extraction algorithms which use \acs{sar} image data \\ collected during the \ac{scale-win22} cruise.\\    (d) Test and evaluate the system performance based on performance metrics.\\    (e) Discuss the performance of the system, drawing conclusions and \\ make recommendations for future improvements.\end{tabular} \\ \hline
\end{tabular}
\caption{Project objectives as defined in the project brief provided by Robyn Verrinder}
\label{tab:intro.projectObjectives.brief}
\end{table}

%====================================================
\section{Scope, limitations and assumptions}
\label{sec:intro.scope}
%====================================================
This project forms a preliminary investigation into developing a \acs{sar} processing pipeline for sea ice parameter extraction. The entire pipeline will be tested, and validated at various stages of processing steps. The implemented validation methods, experiments and system design must be well-documented and repeatable for future development and expansion of the parameter extraction pipeline.

As this project is a preliminary investigation in parameter extraction using \acs{sar} data, an investigation into the application of Antarctic sea ice dispersion and attenuation models to parameter extraction from \acs{sar} data falls outside of this project's scope. This project focuses on the extraction of wave parameters from \acs{sar} data, but the context of this work falls within a larger research effort for sea ice parameter extraction, and this context will be provided as it allows the outputs of the pipeline to be determined.

The following limitations exist for this project:
\begin{itemize}
    \item Wave data used to generate wave spectra was sourced from the NOAA, and had a resolution of 0.25 degrees.
    \item 
\end{itemize}

% The project will focus on implementing a procedure known as the Hasselmann Nonlinear Mapping of an Ocean Wave Spectrum \cite{Hasselmann1991}, to estimate the sea state from a SAR image. Sentinel 1A captures will be used to obtain access to SAR data, and wave models of the region of interest will be used to generate a first-guess spectrum of the sea state. The mathematics described by Hasselmann will be implemented on the first-guess wave spectrum, as well as data obtained from the Sentinel 1A satellite. This project’s scope will only include the extraction of wave parameters from SAR data, but the context of this work falls within a larger research effort for sea ice parameter extraction, and this context will be provided as it allows the outputs of the pipeline to be determined.

% This project will make use of the following data: NOAA wave data from WaveWatch 3 (WW3), and Sentinel-1A SAR data. The limitations of both data are the resolution of the wave data (0.25 degrees), as well as the regions over which Sentinel data can be captured respectively. The ground truth data which will be used is from the directional wave buoy located in 70m deep water near Cape Point at 34.204S, 18.287E. Due to 70m being a good approximation for deep water, wave data from NOAA surrounding this region should serve as a good estimate of the sea state at the wave buoy. With respect to SAR data, the Cape Point region is covered by Sentinel-1A, and archived data is available since Sentinel-1A began operation in 2014.

%====================================================
\section{Plan of development}
\label{sec:intro.planofdevelopment}
%====================================================
This project report begins with an overview of the relevant literature related to this project in Chapter \ref{chap:litReview}, followed by the introduction and explanation of the relevant theory in Chapter \ref{chap:theory}. These two chapters will introduce the pertinent subject matter of this report and provide context for the work undertaken in this project.

Chapter \ref{chap:litReview} aims to frame the context of this work by discussing previous research efforts and highlighting the necessity for this project. In contrast, Chapter \ref{chap:theory} aims to provide a theoretical introduction to the themes of \acs{sar}, along with the associated wave modelling techniques.
% Can improve
After discussing the relevant literature and theory, Chapter \ref{chap:systemDesign} will present the approach taken to address the problem described in Section \ref{sec:intro.problemstatement}, explaining the procedures employed and the methods used for testing. Furthermore, this chapter will detail the context of this work within the broader pipeline design before elaborating on the pipeline development, as well as the methods and functions developed for its implementation.

Subsequently, Chapter \ref{chap:pipelineValidation} will introduce the implementation of the pipeline described in Chapter \ref{chap:systemDesign}. This section will encompass the validation experiments conducted to determine the efficacy of individual sub-systems and functions described in Chapter \ref{chap:systemDesign}.

Chapter \ref{chap:sysValidation} will provide a comprehensive analysis of the entire application and validation of the pipeline for wave parameter extraction from \acs{sar} data. This analysis will follow the design detailed in Chapter \ref{chap:systemDesign}, combining the sub-systems tested in Chapter \ref{chap:pipelineValidation}, and will involve a comparison of output data against the desired testing outputs. Chapter \ref{chap:discussion} will then discuss the results obtained in Chapters \ref{chap:pipelineValidation} and \ref{chap:sysValidation} in terms of the initial scope, limitations, and assumptions before expanding this discussion to the context of the broader literature on the topic. This chapter will then present approaches for future work required in this project. Chapter \ref{chap:conclusion} will then conclude the project and draw final conclusions from the subsequent chapters.

%----------------------------------------------------
%****************************************************
% END
%****************************************************
