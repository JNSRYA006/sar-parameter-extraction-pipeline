% This LaTeX was auto-generated from MATLAB code.
% To make changes, update the MATLAB code and export to LaTeX again.

\sloppy
\epstopdfsetup{outdir=./}
\graphicspath{ {./pipeline_images/} }

\matlabhastoc


\label{T_B8033DED}
\matlabtitle{Wave Parameter Extraction Pipeline}

\begin{par}
\begin{flushleft}
This MATLAB LiveScript serves as a live document to show the implementation of the SAR wave parameter extraction pipeline with additional documentation. Documentation notes are included before code blocks which explain the code which follows.
\end{flushleft}
\end{par}

\matlabtableofcontents{Table of Contents}

\label{H_45278BA7}
\matlabheading{Prerequisites}

\label{H_603A9976}
\matlabheadingtwo{Software Prerequisites}

\label{H_FF806E53}
\matlabheadingthree{SNAP ESA}

\begin{par}
\begin{flushleft}
SNAP is a tool developed by the European Space Agency (ESA) to process SAR data obtained from Sentinel Satellites. SNAP is used to pre-process data in this pipeline and is essential to install.
\end{flushleft}
\end{par}

\begin{par}
\begin{flushleft}
To install SNAP please do the following:
\end{flushleft}
\end{par}

\begin{itemize}
\setlength{\itemsep}{-1ex}
   \item{\begin{flushleft} Download SNAP \href{https://step.esa.int/main/download/snap-download/}{here} for your OS distribution \end{flushleft}}
   \item{\begin{flushleft} Choose only the Sentinel Toolboxes installer \end{flushleft}}
   \item{\begin{flushleft} Install SNAP and follow the onscreen instructions. It is only necessary to install the Sentinel-1 Toolbox \end{flushleft}}
\end{itemize}

\label{H_69BBBAC5}
\matlabheadingtwo{MATLAB Prerequisites}

\begin{par}
\begin{flushleft}
Python 3.8 (Need to check version in MATLAB)
\end{flushleft}
\end{par}

\begin{matlabcode}
pe = pyenv;
pyVersion = pe.Version;
if ~(strcmp(pyVersion, "3.8"))
    pe(Version="3.8")
end
\end{matlabcode}

\begin{par}
\begin{flushleft}
CDS Toolbox
\end{flushleft}
\end{par}

\label{H_E263194C}
\matlabheadingthree{M\_Map}

\begin{par}
\begin{flushleft}
M\_Map is a mapping package for MATLAB which allows data plots on different world maps.
\end{flushleft}
\end{par}

\begin{par}
\begin{flushleft}
To use M\_Map please do the following:
\end{flushleft}
\end{par}

\begin{itemize}
\setlength{\itemsep}{-1ex}
   \item{\begin{flushleft} Download the zipped package \href{https://www.eoas.ubc.ca/~rich/map.html#download}{here} and \end{flushleft}}
   \item{\begin{flushleft} Add the extracted folder to your MATLAB path \end{flushleft}}
\end{itemize}

\begin{par}
\begin{flushleft}
\textit{Note:} A copy of the M\_Map package is included on this git repo in the following path: 
\end{flushleft}
\end{par}

\label{H_ADB0A04B}
\matlabheadingthree{wgrid}


\label{H_7B094A5D}
\matlabheading{Obtain SAR Data}

\begin{par}
\begin{flushleft}
SAR data from Sentinel-1A can be downloaded from \href{https://scihub.copernicus.eu/dhus/#/home}{Copernicus SciHub}. In order to download data, you need to register an account on the site.
\end{flushleft}
\end{par}

\begin{par}
\begin{flushleft}
Once you have registered your account, you can search for SAR data on the site by choosing the appropriate filters and region. The region you search is set by right-clicking and drawing a rectangle, otherwise you can left-click and draw the vertices of any polygon. Recomended filters are:
\end{flushleft}
\end{par}

\begin{enumerate}
\setlength{\itemsep}{-1ex}
   \item{\begin{flushleft} \textbf{Sensing Period: }As desired \end{flushleft}}
   \item{\begin{flushleft} \textbf{Satellite Platform:} S1A \end{flushleft}}
   \item{\begin{flushleft} \textbf{Product Type:} GRD \end{flushleft}}
   \item{\begin{flushleft} \textbf{Polarisation:} As desired \end{flushleft}}
\end{enumerate}

\begin{par}
\begin{flushleft}
After searching with your filters, you will see all the available data within this region. Clicking on a red footprint on the map, shows a preview of these data, and the data can be downloaded from the pop-up. 
\end{flushleft}
\end{par}

\begin{par}
\begin{flushleft}
\textit{Note: }These data are in excess of 1GB, so large amounts of storage space is required. Use an external hard drive if possible.
\end{flushleft}
\end{par}


\label{H_1D14B3F6}
\matlabheading{\textbf{Pre-Process SAR Data}}

\begin{par}
\begin{flushleft}
After downloading SAR data from \href{https://scihub.copernicus.eu/dhus/#/home}{Copernicus SciHub}, open \hyperref[H_FF806E53]{SNAP} and import the data. These data can be viewed as follows:
\end{flushleft}
\end{par}

\begin{enumerate}
\setlength{\itemsep}{-1ex}
   \item{\begin{flushleft} Click the '+' next to the dataset \end{flushleft}}
   \item{\begin{flushleft} Click the '+' next to the \textit{Bands} folder \end{flushleft}}
   \item{\begin{flushleft} Open the desired band to preview the image \end{flushleft}}
\end{enumerate}

\label{H_B7EBD40C}
\matlabheadingtwo{Thermal Noise Calibration}

\begin{par}
\begin{flushleft}
\textbf{This needs to be done first}
\end{flushleft}
\end{par}

\begin{par}
\begin{flushleft}
To remove thermal noise, click \textit{Radar -\textgreater{} Radiometric -\textgreater{} S1 Thermal Noise Removal. }This will bring up a window where the following needs to be done.
\end{flushleft}
\end{par}

\begin{enumerate}
\setlength{\itemsep}{-1ex}
   \item{\begin{flushleft} Ensure the file will save as \textit{BEAM-DIMAP} \end{flushleft}}
   \item{\begin{flushleft} Check that \textit{Remove Thermal Noise} is checked in the \textit{Processing Parameters} window \end{flushleft}}
   \item{\begin{flushleft} Select a specific polarisation (If desired) \end{flushleft}}
   \item{\begin{flushleft} Rename the output file (\textit{Target Product) }if desired \end{flushleft}}
   \item{\begin{flushleft} Check the \textit{Open in SNAP} checkbox \end{flushleft}}
\end{enumerate}

\begin{par}
\begin{flushleft}
After clicking \textit{Run,} SNAP will begin removing thermal noise from the selected data and once complete will open in the \textit{Product Explorer }sidebar.
\end{flushleft}
\end{par}

\label{H_0477ED49}
\matlabheadingtwo{Radiometric Calibration}

\begin{par}
\begin{flushleft}
To radiometrically calibrate the SAR data, ensure that the \textbf{noise calibrated product is selected}, then click \textit{Radar -\textgreater{} Radiometric -\textgreater{} Calibrate }This will bring up a window where the following needs to be done.
\end{flushleft}
\end{par}

\begin{enumerate}
\setlength{\itemsep}{-1ex}
   \item{\begin{flushleft} Ensure the file will save as \textit{BEAM-DIMAP} \end{flushleft}}
   \item{\begin{flushleft} Check that \textbf{only} \textit{Output sigma0 band} is checked in the \textit{Processing Parameters} window \end{flushleft}}
   \item{\begin{flushleft} Select a specific polarisation (If desired) \end{flushleft}}
   \item{\begin{flushleft} Rename the output file (\textit{Target Product) }if desired \end{flushleft}}
   \item{\begin{flushleft} Check the \textit{Open in SNAP} checkbox \end{flushleft}}
\end{enumerate}

\begin{par}
\begin{flushleft}
After clicking \textit{Run,} SNAP will begin radiometric calibration from the selected data and once complete will open in the \textit{Product Explorer }sidebar.
\end{flushleft}
\end{par}

\label{H_C1EC94A9}
\matlabheadingtwo{Export Incidence Angle}

\begin{par}
\begin{flushleft}
In order to access the incidence angle of each pixel in the the SAR data, the following process needs to be followed.
\end{flushleft}
\end{par}

\begin{enumerate}
\setlength{\itemsep}{-1ex}
   \item{\begin{flushleft} Click the '+' next to the \textit{Tie-Point Grids} folder \end{flushleft}}
   \item{\begin{flushleft} Right-click on the \textit{incident\_angle} band \end{flushleft}}
   \item{\begin{flushleft} Select \textit{Band Maths} \end{flushleft}}
   \item{\begin{flushleft} Rename the band to "Incidence\_Angle" and click OK \end{flushleft}}
\end{enumerate}

\label{H_69B1A933}
\matlabheadingtwo{Exporting as NetCDF}

\begin{par}
\begin{flushleft}
In order to export these pre-processed SAR data in a format supported by MATLAB, the following process needs to be followed.
\end{flushleft}
\end{par}

\begin{enumerate}
\setlength{\itemsep}{-1ex}
   \item{\begin{flushleft} Click \textit{File} \end{flushleft}}
   \item{\begin{flushleft} Hover over the \textit{Export} option, and select \textbf{NetCDF4-BEAM} \end{flushleft}}
   \item{\begin{flushleft} If your file explorer is opened, click on \textit{Subset... }on the right hand side of the dialog box \end{flushleft}}
   \item{\begin{flushleft} This will bring up a new window with four different menus. \end{flushleft}}
   \item{\begin{flushleft} You can take a Spatial Subset of the image using either pixel or geographical coordinates in the \textit{Spatial Subset }menu \end{flushleft}}
   \item{\begin{flushleft} Ensure that under the \textit{Band Subset} menu, your desired bands are selected, as well as your created \textit{\textbf{Incidence\_Angle}}\textbf{ band. } \end{flushleft}}
   \item{\begin{flushleft} Ensure that under the \textit{Metadata Subset} menu, all options are selected \end{flushleft}}
   \item{\begin{flushleft} After checking all of these parameters, click OK, and name the file as desired and save it in your desired location \end{flushleft}}
\end{enumerate}

\begin{par}
\begin{flushleft}
Optional
\end{flushleft}
\end{par}


\vspace{1em}

\label{H_A008662F}
\matlabheading{Take transects and subdivide}

\begin{par}
\begin{flushleft}
After exporting the SAR data as a NetCDF file, use the following commands to import the data and metadata into MATLAB.
\end{flushleft}
\end{par}

\begin{matlabcode}
filepath = "D:\UCT\EEE4022S\Data\CPT\smaller_subset_incidence.nc";
% Import data values
ncImport = ncinfo(filepath);
% Update the band to import as required
sarData = ncread(filepath,'Sigma0_VV');
metadata = ncinfo(filepath,'metadata');
incidenceAngle = ncread(filepath,'Incidence_Angle');
\end{matlabcode}


\label{H_82C18072}
\matlabheadingtwo{Define Transect Parameters}

\begin{par}
\begin{flushleft}
Enter the number of desired transects to take, the start point (in pixels), along with the angle at which to take the transects. This angle is calculated from the positive x-axis clockwise.
\end{flushleft}
\end{par}

\begin{matlabcode}
% Enter the number of transects
n = 3;
th = 10;
[transectData, positions] = get512Transects(sarData,1,1,th,n);
\end{matlabcode}

\begin{par}
\begin{flushleft}
The original data, with shown transects can be plotted as follows.
\end{flushleft}
\end{par}

\begin{matlabcode}
figure(1)
imshow(sarData)
hold on;

for i = 1:n
    annotate512Transect(positions(i,1),positions(i,2),i,'red','black',1);
end

title('Sentinel-1A data with transects displayed')
hold off
\end{matlabcode}

\begin{par}
\begin{flushleft}
The individual transects can be plotted as follows.
\end{flushleft}
\end{par}

\begin{matlabcode}
figure(2)
subplot(1,3,1)
imshow(transectData(:,:,1));
title(['Transect 1 at ',num2str(th),' degrees'])
subplot(1,3,2)
imshow(transectData(:,:,2));
title(['Transect 2 at ',num2str(th),' degrees'])
subplot(1,3,3)
imshow(transectData(:,:,3));
title(['Transect 3 at ',num2str(th),' degrees'])
\end{matlabcode}

\begin{par}
\begin{flushleft}
Metadata can be imported using the \texttt{ncinfo} function. This large metadata structure can be slimmed down by filtering using the desired attributes for subsequent functions.
\end{flushleft}
\end{par}

\begin{matlabcode}
% Metadata
%metadata = ncinfo(filepath,'metadata');
req_atributes = ["first_near_lat","first_near_long","first_far_lat","first_far_long","last_near_lat","last_near_long","last_far_lat","last_far_long","centre_lat","centre_lon", "num_output_lines","num_samples_per_line"];
metadata_filtered = filterAttributesNetCDF(metadata.Attributes, req_atributes);
\end{matlabcode}


\label{H_1A16A1DF}
\matlabheading{Metadata Extraction}


\vspace{1em}

\vspace{1em}

\label{H_357F0BDA}
\matlabheading{Generate Wave Spectra}

\begin{par}
\begin{flushleft}
Before generating wave spectra, please source data from \href{https://nomads.ncep.noaa.gov/pub/data/nccf/com/gfs/prod/}{NOAA NCEP here}. Ensure that you select data in the following manner.
\end{flushleft}
\end{par}

\begin{enumerate}
\setlength{\itemsep}{-1ex}
   \item{\begin{flushleft} Select data of the following form: \texttt{\textbf{gdas.date}} \end{flushleft}}
   \item{\begin{flushleft} Select the desired time (Either \texttt{00/}, \texttt{06/}, \texttt{12/}, or \texttt{18/}) \end{flushleft}}
   \item{\begin{flushleft} Select \texttt{wave/} data \end{flushleft}}
   \item{\begin{flushleft} Choose the \texttt{gridded/ }data \end{flushleft}}
   \item{\begin{flushleft} Once completing all of the above, choose any data in the following form: \texttt{gdaswave.}\texttt{\textbf{time}}\texttt{z.}\texttt{\textbf{location}}\texttt{.0p25.f000.grib2, }where the \textbf{bold} values are user decided variables \end{flushleft}}
   \item{\begin{flushleft} After choosing a data set, right-click and copy the link address \end{flushleft}}
\end{enumerate}

\begin{par}
\begin{flushleft}
Parse in your chosen dataset
\end{flushleft}
\end{par}

\begin{matlabcode}
noaaUrl = "https://nomads.ncep.noaa.gov/pub/data/nccf/com/gfs/prod/gdas.20231002/12/wave/gridded/gdaswave.t12z.gsouth.0p25.f000.grib2";
\end{matlabcode}

\begin{par}
\begin{flushleft}
Set the name for the saved downloaded \texttt{.grib2} file and the file will be downloaded and saved.
\end{flushleft}
\end{par}

\begin{matlabcode}
waveDownloadName = "wave_data.grib2";
waveFilePath = downloadNOAAWaveFile(noaaUrl,waveDownloadName);
\end{matlabcode}

\begin{par}
\begin{flushleft}
Set the location of wgib and the downloaded \texttt{.grib2} filepath as well as choosing if you'd like to plot your downloaded data along with the type to plot
\end{flushleft}
\end{par}

\begin{matlabcode}
GribPath = "C:\Users\ryanj\OneDrive - University of Cape Town\4. Fourth Year\Second Semester\EEE4022S\repo\sar-parameter-extraction-pipeline\functions\";
wgrib2Path = "C:\Users\ryanj\Downloads";
waveStruct = getGribStruct(wgrib2Path,GribPath);
noaaPlot = true;
noaaValToPlot = 'direction';
if noaaPlot
    noaaDataPlot('miller',waveStruct,noaaValToPlot);
end
\end{matlabcode}

\begin{par}
\begin{flushleft}
Set the location at which to generate wave spectra for
\end{flushleft}
\end{par}

\begin{matlabcode}
latitude = -34.2;
longitude = 18.6;
[gridLat, gridLon] = createLatLonGrid(latitude, longitude, resolution);
startLat = max(gridLat)
startLon = min(gridLon)
endLat = min(gridLat)
endLon = max(gridLon)
waveVals = getSubsetWaveVals(waveStruct,startLat,startLon,endLat,endLon);
\end{matlabcode}

\label{H_D6296C3F}
\matlabheadingtwo{Generate One-dimensional Wave Spectra}

\begin{par}
\begin{flushleft}
Define and instantiate variables for the one-dimensional wave spectra
\end{flushleft}
\end{par}

\begin{matlabcode}
imageSize = size(sarData,1);
w = linspace(0,2*pi,imageSize)';
f = linspace(0,1,imageSize)';
Hs = waveVals.significantWaveHeight(1,1);
T0 = waveVals.significantWavePeriod(1,1);
w0 = 2*pi./T0;
f0 = 1./T0;
gammaVal = 1.308;
multipleWaveSpectra = false;
if multipleWaveSpectra
    S = generateMultipleJONSWAP(waveVals,gammaVal,w,1);   
else
    S = generateSingleJONSWAP(Hs,w0,gammaVal,w);
end
\end{matlabcode}

\label{H_D690D0F0}
\matlabheadingtwo{Generate Two-dimensional Wave Spectra}

\begin{matlabcode}
[D,theta] = generateDirectionalDistribution(waveVals,w,1);
E = generate2DWaveSpectrum(S,D);
\end{matlabcode}


\vspace{1em}

\label{H_63792B58}
\matlabheading{SAR Spectrum Calculation}


\vspace{1em}

\label{H_B1CC5E95}
\matlabheading{Hasselmann Procedure}


\vspace{1em}

\vspace{1em}
\label{H_893CF502}
\matlabheadingtwo{Generate SAR spectrum of Ocean Waves}


\vspace{1em}
\label{H_F29E6812}
\matlabheadingtwo{Inversion}


\vspace{1em}

\label{H_06FC8CFF}
\matlabheading{Output Parameters}
